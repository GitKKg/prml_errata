\documentclass[12pt,a4paper]{article}

\usepackage[T1]{fontenc}
\usepackage{lmodern}
\usepackage{bm}
\usepackage{amsfonts}
\usepackage{mathtools}
\usepackage[margin=2cm]{geometry}
\usepackage{natbib}
\usepackage{hyperref}
\usepackage{url}

\newcommand{\erratum}[1]{\subsubsection*{#1}}

\begin{document}

\title{PRML Errata}
\author{\href{mailto:yousuketakada@gmail.com}{Yousuke Takada}}
\date{\today}
\maketitle

\section*{Preface}
This report communicates some possible errata for PRML~\citep{Bishop:PRML}
that are not listed in the official errata document~\citep{Svensen:PRML_errata}\footnote{\
The last line but one of the bibliographic information page of the copy of PRML I have
reads ``9 8 7 (corrected at 6th printing 2007).'' So I refer to Version~2 of the errata.}
at the time of this writing.
When specifying the location of an error, I follow the notational conventions adopted by
\citet{Svensen:PRML_errata}.
I have also included in this report some suggestions for improving the readability.

\section*{Corrections}

\erratum{Page~51}
Equation~(1.98):
Following the notation~(1.93), we should write the left hand side of (1.98) as
$\operatorname{H}[X]$ instead of $\operatorname{H}[p]$.

\erratum{Page~80}
Equation~(2.52):
We usually take eigenvectors~$\mathbf{u}_{i}$ to be the columns of $\mathbf{U}$ as in (C.37).
If we follow this convention, Equation~(2.52) and the following text should read
\begin{equation}
\mathbf{y} = \mathbf{U}^{\operatorname{T}} (\mathbf{x} - \bm{\mu})
\label{eq:change_of_variable_x_to_y}
\end{equation}
where $\mathbf{U}$ is a matrix whose columns are given by $\mathbf{u}_{i}$ so that
$\mathbf{U} = \left( \mathbf{u}_{1}, \dots,  \mathbf{u}_{D}\right)$.
From (2.46) it follows that $\mathbf{U}$ is an \emph{orthogonal} matrix, i.e.,
it satisfies $\mathbf{U}^{\operatorname{T}}\mathbf{U} = \mathbf{I}$ and hence also
$\mathbf{U}\mathbf{U}^{\operatorname{T}} = \mathbf{I}$ where $\mathbf{I}$ is the identity matrix.

\erratum{Page~81}
Equations~(2.53) and (2.54):
If we write the change of variable from $\mathbf{x}$ to $\mathbf{y}$ as
\eqref{eq:change_of_variable_x_to_y} instead of (2.52),
the Jacobian matrix~$\mathbf{J} = \left( J_{ij} \right)$ we require is
simply given by $\mathbf{U}$.
Equation~(2.53) should read
\begin{equation}
J_{ij} = \frac{\partial x_i}{\partial y_j} = U_{ij}
\end{equation}
where $U_{ij}$ is the $(i, j)$-th element of the matrix $\mathbf{U}$.
The square of the determinant of the Jacobian matrix~(2.54) can be evaluated as follows.
\begin{equation}
\left| \mathbf{J} \right|^{2} = \left| \mathbf{U} \right|^{2}
= \left| \mathbf{U}^{\operatorname{T}} \right| \left| \mathbf{U} \right|
= \left| \mathbf{U}^{\operatorname{T}} \mathbf{U} \right|
= \left| \mathbf{I} \right| = 1.
\end{equation}

\erratum{Page~81}
Line~$-1$:
Since the determinant of the Jacobian, which is an orthonormal matrix here, can be negative,
we should write $\left| \mathbf{J} \right| = \pm 1$
instead of $\left| \mathbf{J} \right| = 1$.

\erratum{Page~82}
Equation~(2.56):
We should take the absolute value of the determinant for the same reason given above;
the factor~$\left| \mathbf{J} \right|$ should read
$\left| \operatorname{det}\left( \mathbf{J} \right) \right|$.
Note that we cannot write $\left|\left| \mathbf{J} \right|\right|$ to mean
$\left| \operatorname{det}\left( \mathbf{J} \right) \right|$
because it is confusingly similar to the matrix norm~$\left\| \mathbf{J} \right\|$,
which usually refers to the largest singular value of $\mathbf{J}$ \citep{GvL:MC}.
This notational inconsistency has been caused by the abuse of the notation~$|\cdot|$ for
both the absolute value and the matrix determinant.
If we always use $\operatorname{det}(\cdot)$ for the determinant, confusion will not arise
and the notation be consistent.
An alternative solution to this problem would be to explicitly define
\begin{equation}
\left| \mathbf{A} \right| \equiv \left| \operatorname{det}\left( \mathbf{A} \right) \right|
\label{eq:abs_of_matrix_determinant}
\end{equation}
for any square matrix $\mathbf{A}$.
This notation is mostly consistent because we have
$\left| \mathbf{A} \right| = \operatorname{det}\left( \mathbf{A} \right)$ for a positive
semidefinite matrix $\mathbf{A}$ and
most other matrices for which we take determinants are positive (semi)definite in PRML.

\erratum{Page~102}
Equation~(2.155):
Although an interpretation for the parameters of the gamma distribution~(2.146) has been made,
one for the parameters of the Wishart distribution~(2.155) is not given here nor in Exercise~2.45.
In order to give one, let us consider a simple Bayesian inference problem in which,
given a set of $N$ observations~$\mathbf{X} =
\left\{ \mathbf{x}_1, \dots, \mathbf{x}_N \right\}$ for a zero-mean Gaussian random variable,
we infer the covariance matrix~$\bm{\Sigma}$ or, equivalently,
the precision matrix~$\bm{\Lambda} \equiv \bm{\Sigma}^{-1}$.
The likelihood~$p\left(\mathbf{X}\middle|\bm{\Lambda}\right)$
in terms of the precision~$\bm{\Lambda}$ is given by
\begin{equation}
p\left(\mathbf{X}\middle|\bm{\Lambda}\right) =
\prod_{n=1}^{N} p\left(\mathbf{x}_n\middle|\bm{\Lambda}\right) =
\prod_{n=1}^{N} \mathcal{N}\left(\mathbf{x}_n\middle|\mathbf{0}, \bm{\Lambda}^{-1}\right) .
\end{equation}
If we choose the prior~$p\left(\bm{\Lambda}\right)$ over $\bm{\Lambda}$ to be
a Wishart distribution so that
\begin{equation}
p\left(\bm{\Lambda}\right) = \mathcal{W}\left(\bm{\Lambda}\middle|\mathbf{W}_0, \nu_0\right)
\label{eq:wishart_prior}
\end{equation}
our analysis can be simplified because it is the conjugate prior.
If fact, the posterior~$p\left(\bm{\Lambda}\middle|\mathbf{X}\right)$ is given by
\begin{align}
p\left(\bm{\Lambda}\middle|\mathbf{X}\right) &\propto
p\left(\mathbf{X}\middle|\bm{\Lambda}\right) p\left(\bm{\Lambda}\right) \\
&\propto \left|\bm{\Lambda}\right|^{N/2}
\exp\left\{-\frac{1}{2}\sum_{n=1}^{N}\mathbf{x}_n^{\operatorname{T}}\bm{\Lambda}\mathbf{x}_n\right\}
\left|\bm{\Lambda}\right|^{(\nu_0 - D - 1)/2}
\exp\left\{-\frac{1}{2}\operatorname{Tr}\left(\mathbf{W}_0^{-1}\bm{\Lambda}\right)\right\} \\
&= \left|\bm{\Lambda}\right|^{(\nu_N - D - 1)/2}
\exp\left\{-\frac{1}{2}\operatorname{Tr}\left(\mathbf{W}_N^{-1}\bm{\Lambda}\right)\right\}
\end{align}
where
\begin{align}
\nu_N &= \nu_0 + N \\
\mathbf{W}_N^{-1} &=
\mathbf{W}_0^{-1} + \sum_{n=1}^{N}\mathbf{x}_n\mathbf{x}_n^{\operatorname{T}} .
\end{align}
Reinstating the normalization constant, we indeed see that the posterior becomes again
a Wishart distribution of the form
\begin{equation}
p\left(\bm{\Lambda}\middle|\mathbf{X}\right) =
\mathcal{W}\left(\bm{\Lambda}\middle|\mathbf{W}_N, \nu_N\right) .
\end{equation}
This result suggests us how we can interpret the parameters of
the Wishart distribution~(2.155), namely
the scale matrix~$\mathbf{W}$ and the number of degrees of freedom~$\nu$.
Since observing $N$ data points increases the number of degrees of freedom~$\nu$ by $N$,
we can interpret $\nu_0$ in the prior~\eqref{eq:wishart_prior} as
the number of ``effective'' prior observations.
The $N$ observations also contribute $N\bm{\Sigma}_{\text{ML}}$ to
the inverse of the scale matrix~$\mathbf{W}$
where $\bm{\Sigma}_{\text{ML}}$ is the maximum likelihood estimate for the covariance of
the observations given by
\begin{equation}
\bm{\Sigma}_{\text{ML}} = \frac{1}{N} \sum_{n=1}^{N}\mathbf{x}_n\mathbf{x}_n^{\operatorname{T}}
\end{equation}
which suggests an interpretation of $\mathbf{W}$ in terms of
$\bm{\Sigma} \equiv \left(\nu\mathbf{W}\right)^{-1}$, that is,
$\bm{\Sigma}_0$ in the prior~\eqref{eq:wishart_prior} is
the covariance of the ``effective'' prior observations.
Note that this interpretation is in accordance with another observation that
the expectation of $\bm{\Lambda}$ taken over the prior~\eqref{eq:wishart_prior} is indeed given by
$\mathbb{E}\left[\bm{\Lambda}\right] = \nu_0\mathbf{W}_0 = \bm{\Sigma}_0^{-1}$
where we have used (B.80).

\erratum{Page~102}
Line~$-2$:
`Gamma' should read `gamma' (without capitalization).

\erratum{Page~104}
The text after Equation~(2.160):
The Gaussian~$\mathcal{N}\left(\mathbf{x}\middle|\bm{\mu}, \mathbf{\Lambda}\right)$
should read $\mathcal{N}\left(\mathbf{x}\middle|\bm{\mu}, \mathbf{\Lambda}^{-1}\right)$.

\erratum{Page~141}
Equation~(3.13):
The use of the gradient operator is not consistent here.
As in Equation~(2.224), the gradient (of a scalar function) is a column vector
so that Equation~(3.13) should read\footnote{\
I have not got the right typeface for the data vector
$\left( t_1, \dots, t_N \right)^{\operatorname{T}}$.
See `Mathematical notation' in PRML.}
\begin{equation}
\nabla \ln p \left( \mathbf{t} \middle| \mathbf{w}, \beta \right) =
\beta \sum_{n=1}^{N}
\left\{ t_n - \mathbf{w}^{\operatorname{T}} \bm{\phi}\left( \mathbf{x}_n \right) \right\}
\bm{\phi}\left( \mathbf{x}_n \right) .
\label{eq:gradient_of_linear_regression_likelihood}
\end{equation}
Moreover, I would like to also suggest that we should give a definition for the gradient
before we use it or, at least, in an appendix.
Although Appendix~C defines the vector derivative~$\frac{\partial}{\partial \mathbf{x}}$,
which is used interchangeably with the gradient~$\nabla_{\mathbf{x}}$ throughout PRML,
there is no mention of the gradient.
We will come back to this issue later in this report.

\erratum{Page~142}
Equation~(3.14):
The left hand side should be a zero vector~$\mathbf{0}$ instead of a scalar zero~$0$.
Thus, Equation~(3.14) should read
\begin{equation}
\mathbf{0} =
\sum_{n=1}^{N} t_n \bm{\phi}\left( \mathbf{x}_n \right)
- \left( \sum_{n=1}^{N}
\bm{\phi}\left(\mathbf{x}_n\right)\bm{\phi}\left(\mathbf{x}_n\right)^{\operatorname{T}} \right)
\mathbf{w}
\end{equation}
where we have used the gradient of the form~\eqref{eq:gradient_of_linear_regression_likelihood}
instead of (3.13).

\erratum{Page~146}
Equation~(3.31):
The left hand side should be $\mathbf{y}\left(\mathbf{x}, \mathbf{W}\right)$ instead of
$\mathbf{y}\left(\mathbf{x}, \mathbf{w}\right)$.

\erratum{Page~166}
The second paragraph, Line~1:
`Gamma' should read `gamma' (without capitalization).

\erratum{Pages~168--169, and 177}
Equations~(3.88), (3.93), and (3.117) as well as the text before (3.93):
The derivative operators should be partial differentials.
For example, Equation~(3.117) should read
\begin{equation}
\frac{\partial}{\partial \alpha} \ln \left| \mathbf{A} \right| =
\operatorname{Tr}\left( \mathbf{A}^{-1} \frac{\partial}{\partial \alpha} \mathbf{A} \right) .
\end{equation}

\erratum{Page~207}
Equation~(4.92):
On the right hand side, the gradient and the Hessian,
which are in general functions of the parameter~$\mathbf{w}$,
must be evaluated at the previous estimate~$\mathbf{w}^{\text{old}}$ for the parameter.
Thus, Equation~(4.92) should read
\begin{equation}
\mathbf{w}^{\text{new}} = \mathbf{w}^{\text{old}} -
  \left[ \mathbf{H} \left( \mathbf{w}^{\text{old}} \right) \right]^{-1}
  \nabla E \left( \mathbf{w}^{\text{old}} \right)
\end{equation}
where $\mathbf{H}\left(\mathbf{w}\right) \equiv \nabla\nabla E\left(\mathbf{w}\right)$ is
the Hessian matrix whose elements comprise the second derivatives of $E\left(\mathbf{w}\right)$
with respect to the components of $\mathbf{w}$.

\erratum{Page~210}
Equation~(4.110):
The left hand side of (4.110),
which is obtained by taking the gradient of $\nabla_{\mathbf{w}_{j}} E$ given in (4.109)
with respect to $\mathbf{w}_{k}$,
refers to the $(k, j)$-th block of the Hessian, \emph{not} the $(j, k)$-th.
Thus, Equation~(4.110) should read
\begin{equation}
\nabla_{\mathbf{w}_{k}} \nabla_{\mathbf{w}_{j}}
E \left( \mathbf{w}_{1}, \dots, \mathbf{w}_{K} \right)
  = \sum_{n=1}^{N} y_{nj} \left( I_{kj} - y_{nk} \right)
    \bm{\phi}_{n} \bm{\phi}_{n}^{\operatorname{T}}.
\end{equation}
To be clear, we have used the following notation.
If we group all the parameters~$\mathbf{w}_{1}$, $\dots$, $\mathbf{w}_{K}$ into
a column vector
\begin{equation}
\mathbf{w} =
  \begin{pmatrix}
  \mathbf{w}_{1} \\
  \vdots \\
  \mathbf{w}_{K}
  \end{pmatrix}
\end{equation}
the gradient and the Hessian of the error function~$E\left(\mathbf{w}\right)$ with respect to
$\mathbf{w}$ are given by
\begin{equation}
\nabla_{\mathbf{w}} E =
  \begin{pmatrix}
  \nabla_{\mathbf{w}_{1}} E \\
  \vdots \\
  \nabla_{\mathbf{w}_{K}} E
  \end{pmatrix} , \qquad
\nabla_{\mathbf{w}} \nabla_{\mathbf{w}} E =
  \begin{pmatrix}
  \nabla_{\mathbf{w}_{1}} \nabla_{\mathbf{w}_{1}} E &
  \cdots &
  \nabla_{\mathbf{w}_{1}} \nabla_{\mathbf{w}_{K}} E \\
  \vdots & \ddots & \vdots \\
  \nabla_{\mathbf{w}_{K}} \nabla_{\mathbf{w}_{1}} E &
  \cdots &
  \nabla_{\mathbf{w}_{K}} \nabla_{\mathbf{w}_{K}} E
  \end{pmatrix}
\end{equation}
respectively.

\erratum{Page~239}
Figure~5.6:
The eigenvectors~$\mathbf{u}_{1}$ and $\mathbf{u}_{2}$ in Figure~5.6 are unit vectors;
their orientations should be shown as in Figure~2.7.
Or, the scaled vectors should be labeled as
$\lambda_{1}^{-1/2}\mathbf{u}_{1}$ and $\lambda_{2}^{-1/2}\mathbf{u}_{2}$.

\erratum{Page~251}
The second paragraph:
The approximation of the form~(5.84) is usually referred to as
the ``\emph{Gauss--Newton}'' approximation, but not ``\emph{Levenberg--Marquardt}.''
The Levenberg--Marquardt method is a method that improves
the numerical stability of (Gauss--)Newton iterations
by correcting the Hessian matrix so as to be more diagonal dominant~\citep{Press:NR}.

\erratum{Page~275}
The text after Equation~(5.154):
The identity matrix~$\mathbf{I}$ should multiply $\sigma_{k}^{2}(\mathbf{x}_{n})$.

\erratum{Page~277}
Equation~(5.160):
The factor~$L$ should multiply $\sigma_{k}^{2}(\mathbf{x})$ because we have
\begin{align}
s^2(\mathbf{x}) &=
  \mathbb{E}\left[
    \operatorname{Tr}\left\{
      \left( \mathbf{t} - \mathbb{E}\left[ \mathbf{t} \middle| \mathbf{x} \right] \right)
      \left( \mathbf{t} - \mathbb{E}\left[ \mathbf{t} \middle| \mathbf{x} \right]
        \right)^{\operatorname{T}} \right\}
  \middle| \mathbf{x} \right] \\
&= \sum_{k=1}^{K} \pi_{k}(\mathbf{x})
  \operatorname{Tr}\left\{ \sigma_{k}^{2}(\mathbf{x})\mathbf{I} +
    \left( \bm{\mu}_{k}(\mathbf{x}) -
      \mathbb{E}\left[ \mathbf{t} \middle| \mathbf{x} \right] \right)
    \left( \bm{\mu}_{k}(\mathbf{x}) -
      \mathbb{E}\left[ \mathbf{t} \middle| \mathbf{x} \right] \right)^{\operatorname{T}}
  \right\} \\
&= \sum_{k=1}^{K} \pi_{k}(\mathbf{x})
  \left\{ L \sigma_{k}^{2}(\mathbf{x}) +
    \left\| \bm{\mu}_{k}(\mathbf{x}) -
      \mathbb{E}\left[ \mathbf{t} \middle| \mathbf{x} \right] \right\|^{2}
  \right\}
\end{align}
where $L$ is the dimensionality of $\mathbf{t}$.

\erratum{Page~295}
Line~1:
The vector~$\mathbf{x}$ should be a column vector so that
$\mathbf{x} = \left( x_1, x_2 \right)^{\operatorname{T}}$.

\erratum{Page~318}
The text before Equation~(6.93) as well as Equations~(6.93) and (6.94):
The text and the equations should read:
We can evaluate the derivative of $a_n^{\star}$ with respect to $\theta_j$ by differentiating
the relation~(6.84) with respect to $\theta_j$ to give
\begin{equation}
\frac{\partial \mathbf{a}_N^{\star}}{\partial \theta_j}
 = \frac{\partial\mathbf{C}_N}{\partial \theta_j} \left( \mathbf{t}_N - \bm{\sigma}_N \right)
   - \mathbf{C}_N \mathbf{W}_N \frac{\partial \mathbf{a}_N^{\star}}{\partial \theta_j}
   \label{eq:jacobian_of_a_N_wrt_theta_j}
\end{equation}
where the derivatives are Jacobians defined by (C.16) for a vector and
analogously by \eqref{eq:Jacobian_of_matrix_wrt_scalar} for a matrix.
Rearranging \eqref{eq:jacobian_of_a_N_wrt_theta_j} then gives
\begin{equation}
\frac{\partial \mathbf{a}_N^{\star}}{\partial \theta_j}
 = \left( \mathbf{I} + \mathbf{C}_N \mathbf{W}_N \right)^{-1}
   \frac{\partial\mathbf{C}_N}{\partial \theta_j} \left( \mathbf{t}_N - \bm{\sigma}_N \right) .
\end{equation}

\erratum{Page~355}
Equation~(7.117):
The typeface of the vector~$\mathbf{y}$ in (7.117) should be that in (7.110).

\erratum{Page~414}
Figure~8.53, Line~6:
The term~``max-product'' should be ``max-sum.''

\erratum{Page~425}
Equation~(9.3):
The right hand side should be a zero vector~$\mathbf{0}$ instead of a scalar zero~$0$.

\erratum{Page~432}
The text after Equation~(9.13):
It should be noted for clarity that, as the prior~$p(\mathbf{z})$ over $\mathbf{z}$ is
a multinomial distribution~(9.10),
the posterior~$p \left( \mathbf{z} \middle| \mathbf{x} \right)$ over $\mathbf{z}$
given $\mathbf{x}$ is again a multinomial of the form
\begin{equation}
p \left( \mathbf{z} \middle| \mathbf{x} \right) = \prod_{k=1}^{K} \gamma_k^{z_k}
\end{equation}
where we have written $\gamma_k \equiv \gamma(z_k)$,
which can be directly confirmed by inspecting the functional form of the joint distribution
\begin{equation}
p \left( \mathbf{z} \right) p \left( \mathbf{x} \middle| \mathbf{z} \right) =
 \prod_{k=1}^{K} \left\{ \pi_k \,
 \mathcal{N} \left( \mathbf{x} \middle| \bm{\mu}_k, \bm{\Sigma}_k \right) \right\}^{z_k} .
\end{equation}
This observation helps the reader to understand that
evaluating the responsibilities~$\gamma(z_k)$ indeed corresponds to
the E~step of the general EM~algorithm.

\erratum{Page~434}
Equation~(9.15):
Although the official errata~\citep{Svensen:PRML_errata} states that
$\sigma_j$ on the right hand side should be raised to a power of $D$,
the whole right hand side should be raised to $D$ so that Equation~(9.15) should read
\begin{equation}
\mathcal{N}\left( \mathbf{x}_n \middle| \mathbf{x}_n, \sigma_j^2 \mathbf{I} \right)
 = \frac{1}{\left( 2 \pi \sigma_j^2 \right)^{D/2}} .
\end{equation}

\erratum{Page~435}
Equation~(9.16):
The right hand side should be a zero vector~$\mathbf{0}$.

\erratum{Page~465}
Equations~(10.6) and (10.7):
In PRML, Equation (10.6) will be later recognized as ``a negative Kullback-Leibler divergence
between $q_j(\mathbf{Z}_j)$ and $\tilde{p}(\mathbf{X}, \mathbf{Z}_j)$''
(Page~465, Line~$-2$).
However, there is no point in taking a Kullback-Leibler divergence between two probability
distributions over different sets of random variables; such a quantity is undefined.
Moreover, the discussion here seems to be somewhat redundant.
We actually do not have to introduce
the probability~$\tilde{p}(\mathbf{X}, \mathbf{Z}_j)$ other than $q_j^{\star}(\mathbf{Z}_j)$.
Specifically, we can rewrite Equations~(10.6) and (10.7) into
\begin{align}
\mathcal{L}(q) &= \dots \\
&= \int q_j \ln q_j^{\star} \mathrm{d}\mathbf{Z}_j - \int q_j \ln q_j \mathrm{d}\mathbf{Z}_j
 + \text{const} \\
&= - \operatorname{KL}\left( q_j \middle\| q_j^{\star} \right) + \text{const}
\label{eq:variational_optimization_of_lower_bound_wrt_q_j}
\end{align}
where we have defined a new distribution $q_j^{\star}\left(\mathbf{Z}_j\right)$ by the relation
\begin{equation}
\ln q_j^{\star}\left(\mathbf{Z}_j\right) =
\mathbb{E}_{i \neq j}\left[ \ln p\left(\mathbf{X}, \mathbf{Z}\right) \right] + \text{const}.
\label{eq:optimal_solution_for_variational_bayes}
\end{equation}
It directly follows from \eqref{eq:variational_optimization_of_lower_bound_wrt_q_j} that,
since the lower bound $\mathcal{L}(q)$ is the negative Kullback-Leibler divergence between
$q_j(\mathbf{Z}_j)$ and $q_j^{\star}(\mathbf{Z}_j)$ up to some additive constant,
the maximum of $\mathcal{L}(q)$ occurs when $q_j(\mathbf{Z}_j) = q_j^{\star}(\mathbf{Z}_j)$.

\erratum{Page~465}
The text before Equation~(10.8):
The latent variable~$\mathbf{z}_i$ should read $\mathbf{Z}_i$.

\erratum{Page~465}
Line~$-1$:
If we adopt the representation~\eqref{eq:variational_optimization_of_lower_bound_wrt_q_j},
the probability~$\tilde{p}(\mathbf{X}, \mathbf{Z}_j)$ should read
$q_j^{\star}\left(\mathbf{Z}_j\right)$.

\erratum{Page~466}
Line~1:
Again, $\tilde{p}(\mathbf{X}, \mathbf{Z}_j)$ should read
$q_j^{\star}\left(\mathbf{Z}_j\right)$.
The sentence~``Thus we obtain...'' should read
``Thus we see that we have already obtained a general expression for the optimal solution in
\eqref{eq:optimal_solution_for_variational_bayes}.''

\erratum{Page~468}
The text after Equation~(10.16):
The constant term in (10.16) is the \emph{negative} entropy of $p(\mathbf{Z})$.

\erratum{Page~478}
Equation~(10.63):
The additive constant $+1$ on the right hand side should be omitted so that
Equation~(10.63) should read
\begin{equation}
\nu_k = \nu_0 + N_k . \label{eq:reestimation_equation_for_nu}
\end{equation}
A quick check for the correctness of the re-estimation equations would be to consider
a limit of $N \to 0$, in which the effective number of observations~$N_k$ also goes to
zero and the re-estimation equations should reduce to identities.
Equation~(10.63) does not reduces to $\nu_k = \nu_0$, failing the test.
Note that the solution for Exercise~10.13 given by \citet{Svensen:PRML_web_solution} correctly
derives the result~\eqref{eq:reestimation_equation_for_nu}.

\erratum{Page~489}
Equation~(10.107):
The expectations~$\mathbb{E}_{\alpha}\left[ \ln q(\mathbf{w}) \right]_{\mathbf{w}}$ and
$\mathbb{E}\left[ \ln q(\alpha) \right]$ should read
$\mathbb{E}_{\mathbf{w}}\left[ \ln q(\mathbf{w}) \right]$ and
$\mathbb{E}_{\alpha}\left[ \ln q(\alpha) \right]$, respectively,
where the expectation~$\mathbb{E}_{\mathbf{Z}}[\cdot]$ is taken over $q\left(\mathbf{Z}\right)$.

\erratum{Page~489}
Equations~(10.108) through (10.112):
The expectations are notationally inconsistent with (1.36);
they should be of the forms shown in (10.107) or the ones corrected as above.

\erratum{Page~490}
The third paragraph, Line~2:
A comma (,) should be inserted after the ellipsis so that the range of $n$ should read:
$n = 1, \dots, N$.

\erratum{Page~496}
Equation~(10.140):
In order to be consistent with the mathematical notations in PRML,
the differential operator~$d$ in (10.140) should be upright~$\mathrm{d}$.
Moreover, the derivative of $x$ with respect to $x^2$ should be written with parentheses as
$\frac{\mathrm{d}x}{\mathrm{d}\left(x^2\right)}$,
instead of $\frac{\mathrm{d}x}{\mathrm{d}x^2}$, to avoid ambiguity.

\erratum{Page~501}
The text after Equation~(10.162):
The variational parameter $\lambda(\xi)$ is a monotonic function of $\xi$ for $\xi \ge 0$,
but not that its derivative~$\lambda'(\xi)$ is.

\erratum{Page~503}
The text after Equation~(10.168):
A period (.) should be added at the end of the sentence that follows (10.168).

\erratum{Page~512}
Equation~(10.222):
The factor~$\left( 2 \pi v_n \right)^{D/2}$ in the denominator of the right hand side should be
omitted because it has been already included in the Gaussian in (10.213).

\erratum{Page~513}
Equations~(10.223) and (10.224):
The quantities $v^{\text{new}}$ and $\mathbf{m}^{\text{new}}$ in (10.223) and (10.224) are
different from those in (10.217) and (10.218).\footnote{See \citet{Svensen:PRML_errata}
for the errata for Equations~(10.217) and (10.218).}
Thus, we should introduce different notations, say, $v$ and $\mathbf{m}$, with appropriate
definitions.
Specifically, one can rewrite the approximation to the model evidence in the form
\begin{equation}
p(\mathcal{D}) \simeq \left( 2\pi v \right)^{D/2} \exp\left( B/2 \right)
  \prod_{n=1}^{N} \left\{ s_n \left( 2\pi v_n \right)^{-D/2} \right\}
\end{equation}
where
\begin{align}
B &= \frac{\mathbf{m}^{\operatorname{T}}\mathbf{m}}{v}
  - \sum_{n=1}^{N} \frac{\mathbf{m}_n^{\operatorname{T}}\mathbf{m}_n}{v_n} \\
v^{-1} &= \sum_{n=1}^{N} v_n^{-1} \\
v^{-1} \mathbf{m} &= \sum_{n=1}^{N} v_n^{-1} \mathbf{m}_n .
\end{align}

\erratum{Page~515}
Equations~(10.228) and (10.229):
Although \citet{Svensen:PRML_errata} correct (10.228) so that $q^{\backslash b}(\mathbf{x})$ is
a normalized distribution,
we do not need the normalization of $q^{\backslash b}(\mathbf{x})$ here and,
even with this normalization, we cannot ensure that $\hat{p}(\mathbf{x})$ given by (10.229)
is normalized.
Similarly to (10.195), we can proceed with the unnormalized $q^{\backslash b}(\mathbf{x})$ given by
the original (10.228) and, rather than correcting (10.228), we should correct (10.229) so that
\begin{equation}
\hat{p}(\mathbf{x}) \propto q^{\backslash b}(\mathbf{x}) f_b(x_2, x_3) = \dots
\end{equation}
implying that $\hat{p}(\mathbf{x})$ is a normalized distribution.

\erratum{Page~515}
The text after Equation~(10.229):
The new distribution~$q^{\text{new}}(\mathbf{z})$ should read $q^{\text{new}}(\mathbf{x})$.

\erratum{Page~516}
Equation~(10.240):
The subscript~$k$ of the product~$\displaystyle\prod_{k} \dots$ should read $k \neq j$
because we have already removed the term~$\tilde{f}_j(\bm{\theta}_j)$.

\erratum{Page~554}
Equation~(11.72), Line~$-2$:
If we write the expectation $\mathbb{E}_{\mathbf{z}}[\cdot]$ taken over
some given distribution $q(\mathbf{z})$ explicitly as
$\mathbb{E}_{q(\mathbf{z})}\left[\cdot\right]$,
the expectation in the last line but one of (11.72) should read
\begin{equation}
\mathbb{E}_{p_{G}(\mathbf{z})} \left[ \exp\left(-E(\mathbf{z}) + G(\mathbf{z}) \right) \right]
\end{equation}
where we have written the argument~$\mathbf{z}$ for $E(\mathbf{z})$ and $G(\mathbf{z})$
for clarity.

\erratum{Page~556}
Exercise~11.7:
The interval should be $\left[ -\pi/2 , \pi/2 \right]$ instead of $[0, 1]$.

\erratum{Page~557}
Exercise~11.14, Line~2:
The variance should be $\sigma_i^2$ instead of $\sigma_i$.

\erratum{Page~564}
The text after Equation~(12.12):
The derivative we consider here is that with respect to $b_j$
(\emph{not} that with respect to $b_i$).

\erratum{Page~564}
The text after Equation~(12.15):
The zero should be a zero vector so that we have $\mathbf{u}_j = \mathbf{0}$.

\erratum{Page~575}
The third paragraph, Line~5:
The zero vector should be a row vector instead of a column vector so that we have
$\mathrm{v}^{\operatorname{T}}\mathbf{U} = \mathbf{0}^{\operatorname{T}}$.
Or, the both sides are transposed to give $\mathbf{U}^{\operatorname{T}}\mathbf{v} = \mathbf{0}$.

\erratum{Page~578}
Equation~(12.53):
As stated in the text preceding (12.53),
we should substitute $\bm{\mu} = \bar{\mathbf{x}}$ into (12.53).

\erratum{Page~578}
The text before Equation~(12.56):
For the maximization with respect to $\mathbf{W}$, we use (C.25) and (C.27) instead of (C.24).

\erratum{Page~579}
Line~5:
The eigendecomposition requires $O(D^3)$ computation\emph{s}.

\erratum{Page~599}
Exercise~12.1, Line~$-1$:
The quantity $\lambda_{M + 1}$ is an eigenvalue (not an eigenvector).

\erratum{Page~602}
Exercise~12.25, Line~2:
The latent space distribution should read
$p(\mathbf{z}) = \mathcal{N}\left( \mathbf{z} \middle| \mathbf{0}, \mathbf{I}\right)$.

\erratum{Page~610}
The first paragraph, Line~$-5$:
The text~``our predictions ...'' should read: ``our predictions for $\mathbf{x}_{n + 1}$
depend on all the previous observations.''

\erratum{Page~620}
The second paragraph and the following (unlabeled) equation:
The last sentence before the equation and the equation each should terminate with a period (.).

\erratum{Page~621}
Figures~13.12 and 13.13:
It should be clarified that, similarly to $\alpha(z_{nk})$ and $\beta(z_{nk})$,
the notation~$p\left(\mathbf{x}_n \middle| z_{nk}\right)$ is used to denote
the value of $p\left(\mathbf{x}_n \middle| \mathbf{z}_{n}\right)$ when $z_{nk} = 1$.

\erratum{Page~622}
The second paragraph, Line~$-1$:
``we see'' should be omitted.

\erratum{Page~623}
The first paragraph, Line~$-2$:
$z_{nk}$ should read $z_{n - 1, k}$.

\erratum{Page~631}
Equation~(13.73):
The equation should read
\begin{equation}
\sum_{r=1}^{R} \ln \left\{
  \frac{p\left(\mathbf{X}_{r}\middle|\bm{\theta}_{m_r}\right)p(m_r)}{
    \sum_{l=1}^{M}p\left(\mathbf{X}_{r}\middle|\bm{\theta}_{l}\right)p(l)} \right\} .
\end{equation}

\erratum{Page~637}
Equations~(13.81), (13.82), and (13.83):
The distribution~(13.81) over $\mathbf{w}$ should read
\begin{equation}
p(\mathbf{w}) = \mathcal{N}\left(\mathbf{w}\middle|\mathbf{0}, \bm{\Gamma}\right)
\end{equation}
and so on.

\erratum{Page~638}
The first paragraph, Line~2:
``conditional on'' should read ``conditioned on.''

\erratum{Page~641}
The text after Equation~(13.103):
The form of the Gaussian is unclear.
Since a multivariate Gaussian is usually defined over a column vector,
we should construct a column vector from the concerned random variables to clearly define the mean
and the covariance.
Specifically, the text should read for example:
\dots, we see that $\xi(\mathbf{z}_{n-1}, \mathbf{z}_{n})$ is a Gaussian of the form
\begin{equation}
\xi(\mathbf{z}_{n-1}, \mathbf{z}_{n}) =
  \mathcal{N}\left(
    \begin{pmatrix}
      \mathbf{z}_{n-1} \\
      \mathbf{z}_{n}
    \end{pmatrix}
  \middle|
    \begin{pmatrix}
    \hat{\bm{\mu}}_{n-1} \\
    \hat{\bm{\mu}}_{n}
    \end{pmatrix},
    \begin{pmatrix}
      \hat{\mathbf{V}}_{n-1} & \hat{\mathbf{V}}_{n-1, n} \\
      \hat{\mathbf{V}}_{n-1, n}^{\operatorname{T}} & \hat{\mathbf{V}}_{n}
    \end{pmatrix}
  \right)
\end{equation}
where the mean $\hat{\bm{\mu}}_{n}$ and the covariance $\hat{\mathbf{V}}_{n}$ of
$\mathbf{z}_{n}$ are given by (13.100) and (13.101), respectively;
and the covariance $\hat{\mathbf{V}}_{n-1, n}$ between $\mathbf{z}_{n-1}$ and $\mathbf{z}_{n}$ 
is given by
\begin{equation}
\hat{\mathbf{V}}_{n-1, n}
  = \operatorname{cov}\left[\mathbf{z}_{n-1}, \mathbf{z}_{n}\right]
  = \mathbf{J}_{n-1}\hat{\mathbf{V}}_{n} .
\end{equation}

\erratum{Pages~642 and 643}
Equation~(13.109) and the following equations:
If we follow the notation in Chapter~9, the typeface of the $Q$ function should be $\mathcal{Q}$.

\erratum{Page~642}
Equation~(13.109):
If we follow the notation for a conditional expectation given by (1.37),
Equation~(13.109) should read
\begin{align}
\mathcal{Q}\left(\bm{\theta}, \bm{\theta}^{\text{old}}\right)
  &= \mathbb{E}_{\mathbf{Z}} \left[
       \ln p \left( \mathbf{X}, \mathbf{Z} \middle| \bm{\theta} \right) \middle|
       \mathbf{X}, \bm{\theta}^{\text{old}} \right] \\
  &= \int \! \mathrm{d}\mathbf{Z} \;
       p \left( \mathbf{Z} \middle| \mathbf{X}, \bm{\theta}^{\text{old}}\right)
       \ln p \left( \mathbf{X}, \mathbf{Z} \middle| \bm{\theta} \right)
\end{align}
which corresponds to (9.30).

\erratum{Page~643}
Equation~(13.111):
$\mathbf{V}_{0}^{\text{new}}$ should read $\mathbf{P}_{0}^{\text{new}}$.
\citet{Svensen:PRML_errata} have failed to mention (13.111).

\erratum{Page~643}
Equation~(13.114): The size of the opening curly brace `$\{$' should match
that of the closing curly brace `$\}$'.

\erratum{Page~647}
Figure~13.23, Line $-1$:
$p(\mathbf{x}_{n+1}|\mathbf{z}_{n+1}^{(l)})$ should read $p(\mathbf{x}_{n+1}|{z}_{n+1}^{(l)})$.

\erratum{Page~649}
Exercise~13.14, Line 1: (8.67) should be (8.64).

\erratum{Page~658}
Figure~14.1, the equation below:
The subscript of the summation in the right hand side should read $m = 1$.

\erratum{Page~668}
Equation~(14.37):
The arguments of the probability are notationally inconsistent with those of (14.34), (14.35),
and (14.36).
Specifically, the conditioning on $\bm{\phi}_n$ should read that on $t_n$ and
the probability~$p(k|\dots)$ be the value of $p(\mathbf{z}_{n}|\dots)$ when $z_{nk} = 1$,
which we write $p(z_{nk} = 1|\dots)$.
Moreover, strictly speaking,
the old parameters~$\pi_k, \mathbf{w}_k, \beta$ should read
$\pi_k^{\text{old}}, \mathbf{w}_k^{\text{old}}, \beta^{\text{old}} \in \bm{\theta}^{\text{old}}$.
In order to solve these problems, we should rewrite Equation~(14.37) as, for example,
\begin{equation}
\gamma_{nk} = \mathbb{E}\left[ z_{nk} \middle| t_n, \bm{\theta}^{\text{old}} \right]
\end{equation}
where we have written the conditioning in the expectation explicitly and
the expectation is given by
\begin{equation}
\mathbb{E}\left[ z_{nk} \middle| t_n, \bm{\theta} \right]
  = p\left( z_{nk} = 1 \middle| t_n, \bm{\theta} \right)
  = \frac{\pi_k \,
      \mathcal{N}\left(t_n\middle|\mathbf{w}_k^{\operatorname{T}}\bm{\phi}_n, \beta^{-1}\right)}{
    \sum_j \pi_j \,
      \mathcal{N}\left(t_n\middle|\mathbf{w}_j^{\operatorname{T}}\bm{\phi}_n, \beta^{-1}\right)}.
\end{equation}

\erratum{Page~668}
The unlabeled equation between (14.37) and (14.38):
If we write the implicit conditioning in the expectation explicitly
(similarly to the above equations), the unlabeled equation should read
\begin{align}
\mathcal{Q}\left(\bm{\theta}, \bm{\theta}^{\text{old}}\right)
  &= \mathbb{E}_{\mathbf{Z}}\left[ \ln p(\mathbf{t}, \mathbf{Z}|\bm{\theta}) \middle|
       \mathbf{t}, \bm{\theta}^{\text{old}} \right] \\
  &= ...
\end{align}
where we have again used the typeface~$\mathcal{Q}$ for the $Q$ function so that the notation
is consistent with that of Chapter~9.

\erratum{Page~669}
Equations~(14.40) and (14.41): The left hand sides should read a zero vector~$\mathbf{0}$.

\erratum{Page~669}
Equation~(14.41): $\bm{\Phi}$ is undefined.
The text following (14.41) should read:
\dots where $\mathbf{R}_k = \operatorname{diag}(\gamma_{nk})$ is
a diagonal matrix of size $N \times N$ and
$\bm{\Phi} = \left( \bm{\phi}_1, \dots, \bm{\phi}_N \right)^{\operatorname{T}}$ is
an $N \times M$ matrix.
Here, $N$ is the size of the data set and
$M$ is the dimensionality of the feature vectors $\bm{\phi}_n$.

\erratum{Page~669}
Equation~(14.43): `$+\text{const}$' should be added to the right hand side.

\erratum{Page~671}
The text after Equation~(14.46):
The text should read: \dots where we have omitted the dependence on $\{\bm{\phi}_n\}$ and
defined $y_{nk} = \dots$.
Or, $\bm{\phi}$ should have been omitted from the left hand side of (14.45).

\erratum{Page~671}
Equation~(14.48):
The notation should be corrected similarly to the above erratum regarding (14.37).

\erratum{Page~671}
Equation~(14.49):
The notation should be corrected similarly to the above erratum regarding the unlabeled equation
between (14.37) and (14.38).

\erratum{Page~672}
Equation~(14.52):
The negation should be removed so that $\mathbf{H}_k \equiv \nabla_k \nabla_k \mathcal{Q}$ where
\begin{equation}
\nabla_k \nabla_k \mathcal{Q}
   = - \sum_{n=1}^{N} \gamma_{nk} y_{nk} (1 - y_{nk}) \bm{\phi}_n \bm{\phi}_n^{\operatorname{T}}.
\end{equation}

\erratum{Page~674}
Exercise~14.1, Line~1:
``of'' should be inserted after ``set.''

\erratum{Page~686}
Line~$-3$:
The comma in the first inline math should be removed so that the product should read:
$m \times (m-1) \times \dots \times 2 \times 1$.

\erratum{Page~687}
Equation~(B.25):
The differential operator~$d$ should be upright~$\mathrm{d}$.

\erratum{Page~688}
Line~1:
`Gamma' should read `gamma' (without capitalization).

\erratum{Page~689}
Line~1: `positive-definite' should read `positive definite' (without hyphenation).

\erratum{Page~689}
Equation~(B.49):
$\mathbf{x}$ in the right hand side should read $\mathbf{x}_{a}$.

\erratum{Page~690}
Equation~(B.54):
When identifying the functional form of this distribution in,
e.g., the posterior distribution for the Gaussian mixture model of Section~9.2,
I have found it helpful to write the distribution in terms of unnormalized probabilities
$\tilde{\mu}_k \geq 0$ so that
\begin{equation}
\operatorname{Mult}\left(\mathbf{x}\middle|\tilde{\bm{\mu}}\right) =
  C(\tilde{\bm{\mu}}) \prod_{k=1}^{K} {\tilde{\mu}_k}^{x_k}
\end{equation}
where we have defined the normalization constant as
\begin{equation}
{C(\tilde{\bm{\mu}})}^{-1} = \sum_{k=1}^{K} \tilde{\mu}_k .
\end{equation}
and, thus, we have $\mu_k = C(\tilde{\bm{\mu}}) \tilde{\mu}_k$.

\erratum{Page~692}
Equation~(B.68):
This form of multivariate Student's t-distribution is derived in Section~2.3.7
by marginalizing over a gamma distributed scalar variable $\eta$ in (2.161),
but \emph{not} by marginalizing over the precision matrix that is generated from
the conjugate Wishart distribution, which results in a distribution of the form
\begin{equation}
p\left(\mathbf{x}\middle|\bm{\mu}, \mathbf{W}, \nu\right) =
  \int \mathcal{N}\left(\mathbf{x}\middle|\bm{\mu}, \bm{\Lambda}^{-1}\right)
    \mathcal{W}\left(\bm{\Lambda}\middle|\mathbf{W}, \nu\right) \mathrm{d}\bm{\Lambda} .
\label{eq:multivariate_t_marginalized_over_precision}
\end{equation}
I am not sure how we can marginalize over the precision matrix and, moreover,
whether the marginalized distribution \eqref{eq:multivariate_t_marginalized_over_precision} is
equivalent to (B.68).
Some appropriate citation is needed if such details are omitted from the text.

\erratum{Page~693}
Equations~(B.78) through (B.82):
Some appropriate citation is needed for the Wishart distribution
because it has been introduced in Section~2.3.6 without any proof for
the normalization constant as well as other statistics.

\erratum{Page~693}
Line~$-1$: $b = 1/2W$ should read $b = 1/(2W)$ for clarity.

\erratum{Page~696}
Equation~(C.5):
Replacing $\mathbf{B}^{\operatorname{T}}$ with $\mathbf{A}$, we obtain a more general identity
\begin{equation}
\left( \mathbf{P}^{-1} + \mathbf{A}\mathbf{R}^{-1}\mathbf{B} \right)^{-1}
\mathbf{A}\mathbf{R}^{-1}
=
\mathbf{P}\mathbf{A}
\left( \mathbf{B}\mathbf{P}\mathbf{A} + \mathbf{R} \right)^{-1}
\label{eq:general_push_through_identity}
\end{equation}
which is necessary to show the \emph{push-through identity}~(C.6) and also
the determinant identity~(C.14).
As suggested in the text,
the above identity~\eqref{eq:general_push_through_identity} can be
directly verified by right multiplying both sides by
$\left( \mathbf{B}\mathbf{P}\mathbf{A} + \mathbf{R} \right)$.
However, I would prefer to prove
the general push-through identity~\eqref{eq:general_push_through_identity}
together with the Woodbury identity~(C.7) in terms of the inverse of a partitioned matrix,
which we have already seen in Section~2.3.1.
To this end, we first introduce a square matrix $\mathbf{M}$ that is partitioned into
four submatrices so that
\begin{equation}
\mathbf{M} =
\begin{pmatrix}
\mathbf{A} & \mathbf{B} \\
\mathbf{C} & \mathbf{D}
\end{pmatrix}
\end{equation}
where $\mathbf{A}$ and $\mathbf{D}$ are square (but not necessarily the same dimension)
and then note that $\mathbf{M}$ can be block diagonalized as
\begin{equation}
\begin{pmatrix}
\mathbf{I} & \mathbf{O} \\
-\mathbf{C}\mathbf{A}^{-1} & \mathbf{I}
\end{pmatrix}
\begin{pmatrix}
\mathbf{A} & \mathbf{B} \\
\mathbf{C} & \mathbf{D}
\end{pmatrix}
\begin{pmatrix}
\mathbf{I} & -\mathbf{A}^{-1}\mathbf{B} \\
\mathbf{O} & \mathbf{I}
\end{pmatrix}
=
\begin{pmatrix}
\mathbf{A} & \mathbf{O} \\
\mathbf{O} & \mathbf{M}/\mathbf{A}
\end{pmatrix}
\label{eq:block_diagonalization_of_partitioned_matrix_with_Schur_complement_wrt_A}
\end{equation}
or
\begin{equation}
\begin{pmatrix}
\mathbf{I} & -\mathbf{B}\mathbf{D}^{-1} \\
\mathbf{O} & \mathbf{I}
\end{pmatrix}
\begin{pmatrix}
\mathbf{A} & \mathbf{B} \\
\mathbf{C} & \mathbf{D}
\end{pmatrix}
\begin{pmatrix}
\mathbf{I} & \mathbf{O} \\
-\mathbf{D}^{-1}\mathbf{C} & \mathbf{I}
\end{pmatrix}
=
\begin{pmatrix}
\mathbf{M}/\mathbf{D} & \mathbf{O} \\
\mathbf{O} & \mathbf{D}
\end{pmatrix}
\label{eq:block_diagonalization_of_partitioned_matrix_with_Schur_complement_wrt_D}
\end{equation}
if $\mathbf{A}$ or $\mathbf{D}$ is nonsingular, respectively,
where we have written the Schur complement of $\mathbf{M}$ with respect to
$\mathbf{A}$ or $\mathbf{D}$ as
\begin{equation}
\mathbf{M}/\mathbf{A} \equiv \mathbf{D} - \mathbf{C}\mathbf{A}^{-1}\mathbf{B}
\label{eq:Schur_complement_of_M_wrt_A}
\end{equation}
or
\begin{equation}
\mathbf{M}/\mathbf{D} \equiv \mathbf{A} - \mathbf{B}\mathbf{D}^{-1}\mathbf{C}
\label{eq:Schur_complement_of_M_wrt_D}
\end{equation}
respectively.\footnote{\
Note that the notation for the Schur complement is chosen to suggest that
it has a flavor of division~\citep{Minka:OldNewMatrixAlgebra}.
In fact, taking the determinant on both sides of
\eqref{eq:block_diagonalization_of_partitioned_matrix_with_Schur_complement_wrt_A} and
\eqref{eq:block_diagonalization_of_partitioned_matrix_with_Schur_complement_wrt_D},
we have from the definition of the determinant~(C.10) that
\begin{align}
\operatorname{det}(\mathbf{M}) &=
\operatorname{det}(\mathbf{A}) \operatorname{det}(\mathbf{M}/\mathbf{A}) \\
\operatorname{det}(\mathbf{M}) &=
\operatorname{det}(\mathbf{D}) \operatorname{det}(\mathbf{M}/\mathbf{D}) .
\end{align}}
The above block diagonalization
identities~\eqref{eq:block_diagonalization_of_partitioned_matrix_with_Schur_complement_wrt_A}
and \eqref{eq:block_diagonalization_of_partitioned_matrix_with_Schur_complement_wrt_D} yield
two versions of the inverse partitioned matrix~$\mathbf{M}^{-1}$, i.e.,
\begin{align}
\begin{pmatrix}
\mathbf{A} & \mathbf{B} \\
\mathbf{C} & \mathbf{D}
\end{pmatrix}^{-1}
&=
\begin{pmatrix}
\mathbf{I} & -\mathbf{A}^{-1}\mathbf{B} \\
\mathbf{O} & \mathbf{I}
\end{pmatrix}
\begin{pmatrix}
\mathbf{A}^{-1} & \mathbf{O} \\
\mathbf{O} & \left(\mathbf{M}/\mathbf{A}\right)^{-1}
\end{pmatrix}
\begin{pmatrix}
\mathbf{I} & \mathbf{O} \\
-\mathbf{C}\mathbf{A}^{-1} & \mathbf{I}
\end{pmatrix} \\
&=
\begin{pmatrix}
\mathbf{A}^{-1} +
\mathbf{A}^{-1}\mathbf{B} \left(\mathbf{M}/\mathbf{A}\right)^{-1} \mathbf{C}\mathbf{A}^{-1} &
-\mathbf{A}^{-1}\mathbf{B} \left(\mathbf{M}/\mathbf{A}\right)^{-1} \\
-\left(\mathbf{M}/\mathbf{A}\right)^{-1} \mathbf{C}\mathbf{A}^{-1} &
\left(\mathbf{M}/\mathbf{A}\right)^{-1}
\end{pmatrix}
\end{align}
and
\begin{align}
\begin{pmatrix}
\mathbf{A} & \mathbf{B} \\
\mathbf{C} & \mathbf{D}
\end{pmatrix}^{-1}
&=
\begin{pmatrix}
\mathbf{I} & \mathbf{O} \\
-\mathbf{D}^{-1}\mathbf{C} & \mathbf{I}
\end{pmatrix}
\begin{pmatrix}
\left(\mathbf{M}/\mathbf{D}\right)^{-1} & \mathbf{O} \\
\mathbf{O} & \mathbf{D}^{-1}
\end{pmatrix}
\begin{pmatrix}
\mathbf{I} & -\mathbf{B}\mathbf{D}^{-1} \\
\mathbf{O} & \mathbf{I}
\end{pmatrix} \\
&=
\begin{pmatrix}
\left(\mathbf{M}/\mathbf{D}\right)^{-1} &
-\left(\mathbf{M}/\mathbf{D}\right)^{-1} \mathbf{B}\mathbf{D}^{-1} \\
-\mathbf{D}^{-1}\mathbf{C} \left(\mathbf{M}/\mathbf{D}\right)^{-1} &
\mathbf{D}^{-1} +
\mathbf{D}^{-1}\mathbf{C} \left(\mathbf{M}/\mathbf{D}\right)^{-1} \mathbf{B}\mathbf{D}^{-1}
\end{pmatrix}
\end{align}
respectively.
Equating the right hand sides, we have, e.g.,
\begin{equation}
\left(\mathbf{M}/\mathbf{D}\right)^{-1} =
\mathbf{A}^{-1} +
\mathbf{A}^{-1}\mathbf{B} \left(\mathbf{M}/\mathbf{A}\right)^{-1} \mathbf{C}\mathbf{A}^{-1}
\end{equation}
and
\begin{equation}
-\left(\mathbf{M}/\mathbf{A}\right)^{-1} \mathbf{C}\mathbf{A}^{-1} =
-\mathbf{D}^{-1}\mathbf{C} \left(\mathbf{M}/\mathbf{D}\right)^{-1} .
\end{equation}
Substituting \eqref{eq:Schur_complement_of_M_wrt_A} and \eqref{eq:Schur_complement_of_M_wrt_D}
into both sides and
replacing $\mathbf{D}$ with $-\mathbf{D}$,
we finally have
\begin{equation}
\left(\mathbf{A} + \mathbf{B}\mathbf{D}^{-1}\mathbf{C}\right)^{-1} =
\mathbf{A}^{-1} -
\mathbf{A}^{-1}\mathbf{B}
\left(\mathbf{D} + \mathbf{C}\mathbf{A}^{-1}\mathbf{B}\right)^{-1}
\mathbf{C}\mathbf{A}^{-1}
\end{equation}
and
\begin{equation}
\left(\mathbf{D} + \mathbf{C}\mathbf{A}^{-1}\mathbf{B}\right)^{-1} \mathbf{C}\mathbf{A}^{-1} =
\mathbf{D}^{-1}\mathbf{C} \left(\mathbf{A} + \mathbf{B}\mathbf{D}^{-1}\mathbf{C}\right)^{-1}
\end{equation}
which are equivalent to (C.7) and \eqref{eq:general_push_through_identity}, respectively.

\erratum{Page~697}
Equation~(C.17):
It is clear that the definition~(C.17) of the derivative of a scalar with respect to a vector
contradicts (C.16) and (C.18).
The vector derivative of the form~(C.17) is called the \emph{gradient}
whereas (C.18) is called the \emph{Jacobian}~\citep{Minka:OldNewMatrixAlgebra}.
We should use a different notation, say, $\nabla$ for the gradient to avoid ambiguity.
More specifically, given a vector function~$\mathbf{y}(\mathbf{x}) =
\left(y_1(\mathbf{x}), \dots, y_M(\mathbf{x})\right)^{\operatorname{T}}$
where $\mathbf{x} = \left(x_1, \dots, x_D \right)^{\operatorname{T}}$,
we write the gradient of $\mathbf{y}(\mathbf{x})$ with respect to $\mathbf{x}$ as
\begin{equation}
\nabla_{\mathbf{x}} \mathbf{y} = \left( \frac{\partial y_j}{\partial x_i} \right) =
\begin{pmatrix}
\frac{\partial y_1}{\partial x_1} & \hdots & \frac{\partial y_M}{\partial x_1} \\
\vdots & \ddots & \vdots \\
\frac{\partial y_1}{\partial x_D} & \hdots & \frac{\partial y_M}{\partial x_D}
\end{pmatrix} .
\label{eq:gradient_of_vector_wrt_vector}
\end{equation}
As a special case, we see that the gradient of a scalar function~$y(\mathbf{x})$ with respect to
a column vector~$\mathbf{x}$ is again a column vector of the same dimension,
corresponding to the right hand side of (C.17), i.e.,
\begin{equation}
\nabla_{\mathbf{x}} y = \left( \frac{\partial y}{\partial x_i} \right) =
\begin{pmatrix}
\frac{\partial y}{\partial x_1} \\
\vdots  \\
\frac{\partial y}{\partial x_D}
\end{pmatrix} .
\end{equation}
Note also that the right hand side of the definition of
the gradient~\eqref{eq:gradient_of_vector_wrt_vector} is identical to
the transpose of the Jacobian matrix~$\partial\mathbf{y}/\partial\mathbf{x} =
\left( \partial y_i/\partial x_j \right)$
so that $\nabla_{\mathbf{x}} \mathbf{y} =
\left( \partial\mathbf{y}/\partial\mathbf{x} \right)^{\operatorname{T}}$,
as a consequence of which the chain rule for the gradient is such that the intermediate gradients
are built up ``towards the left,'' i.e.,
\begin{equation}
\nabla_{\mathbf{x}} \mathbf{z}(\mathbf{y}) =
\left(
  \frac{\partial\mathbf{z}}{\partial\mathbf{y}}
  \frac{\partial\mathbf{y}}{\partial\mathbf{x}}
\right)^{\operatorname{T}} =
\nabla_\mathbf{x} \mathbf{y} \nabla_\mathbf{y} \mathbf{z}.
\label{eq:chain_rule_for_gradient}
\end{equation}
Since the chain rule~\eqref{eq:chain_rule_for_gradient} is handy
when we compute the gradients of composite functions,
I would suggest that it should also be pointed out in
the ``(Vector and) Matrix Derivatives'' section
of Appendix~C.\footnote{At this point,
one might wonder why we use two different forms of vector derivative that are identical
up to the transposed layout, i.e.,
the gradient~$\nabla_{\mathbf{x}} \mathbf{y}$ and
the Jacobian~$\partial\mathbf{y}/\partial\mathbf{x}$.
I would like to quote what \citet{Minka:OldNewMatrixAlgebra} says regarding this question:
The Jacobian is useful in calculus while the gradient is useful in optimization.}

\erratum{Page~697}
Equation~(C.19):
Following the gradient notation~\eqref{eq:gradient_of_vector_wrt_vector}, (C.19) should read
\begin{equation}
\nabla \left\{ \mathbf{x}^{\operatorname{T}} \mathbf{a} \right\} = 
\nabla \left\{ \mathbf{a}^{\operatorname{T}} \mathbf{x} \right\} = 
\mathbf{a}
\end{equation}
where we have omitted the subscript $\mathbf{x}$ in what should be $\nabla_{\mathbf{x}}$.
Some other useful identities I would suggest to include are
\begin{align}
\nabla \left\{ \mathbf{x}^{\operatorname{T}} \mathbf{A} \mathbf{x} \right\} =
\nabla \operatorname{Tr}\left(\mathbf{x}\mathbf{x}^{\operatorname{T}}\mathbf{A}\right) &=
\left( \mathbf{A}^{\operatorname{T}} + \mathbf{A} \right)\mathbf{x}
\label{eq:gradient_of_quadratic_form} \\
\nabla \left\{ \mathbf{B}\mathbf{x} \right\} & = \mathbf{B}^{\operatorname{T}}
\label{eq:gradient_of_matrix_vector_product} \\
\nabla \left\{ \varphi\mathbf{y} \right\} &=
\nabla\varphi \mathbf{y}^{\operatorname{T}} + \varphi \nabla\mathbf{y}
\label{eq:gradient_of_scalar_vector_product}
\end{align}
where matrices $\mathbf{A}$ and $\mathbf{B}$ are constants.
Note that $\mathbf{x}^{\operatorname{T}} \mathbf{A} \mathbf{x}$ in
\eqref{eq:gradient_of_quadratic_form} is a quadratic form and thus
the square matrix $\mathbf{A}$ is usually taken to be symmetric so that
$\mathbf{A} = \mathbf{A}^{\operatorname{T}}$, in which case we have
\begin{equation}
\nabla \left\{ \mathbf{x}^{\operatorname{T}} \mathbf{A} \mathbf{x} \right\} =
2\mathbf{A} \mathbf{x} \label{eq:gradient_of_quadratic_form_symmetric} .
\end{equation}
Substituting $\mathbf{A} = \mathbf{I}$ gives
\begin{equation}
\nabla \left\| \mathbf{x} \right\|^{2} = 2\mathbf{x} \label{eq:gradiet_of_norm}
\end{equation}
where $\left\| \mathbf{x} \right\| = \sqrt{\mathbf{x}^{\operatorname{T}}\mathbf{x}}$ is
the norm of $\mathbf{x}$.
We make use of the above identity~\eqref{eq:gradiet_of_norm} when, e.g., we take the gradient of
a sum-of-squares error function of the form (3.12),
which can be expressed in terms of the design matrix~$\bm{\Phi}$ given by (3.16) as
\begin{equation}
E(\mathbf{w}) = \frac{1}{2} \left\| \mathbf{t} - \bm{\Phi}\mathbf{w} \right\|^2
\label{eq:sum_of_squares_error} .
\end{equation}
Taking the gradient of \eqref{eq:sum_of_squares_error} with respect to $\mathbf{w}$, we have
\begin{equation}
\nabla_{\mathbf{w}} E(\mathbf{w}) =
-\bm{\Phi}^{\operatorname{T}} \left( \mathbf{t} - \bm{\Phi}\mathbf{w} \right)
\end{equation}
where we have used the identity~\eqref{eq:gradiet_of_norm} together with
the chain rule~\eqref{eq:chain_rule_for_gradient} and
the identity~\eqref{eq:gradient_of_matrix_vector_product}.
The same result can also be obtained by first expanding the square norm in
\eqref{eq:sum_of_squares_error} and then differentiating it using the gradient identities
given above.
We use the identity~\eqref{eq:gradient_of_scalar_vector_product}
when, e.g., we evaluate the Hessian~(5.83) of a nonlinear sum-of-squares error function
such as (5.82), which takes the form
\begin{equation}
J = \frac{1}{2} \sum_{n} \varepsilon_n^{2}.
\end{equation}
The gradient and the Hessian of $J$ can be evaluated as
\begin{align}
\nabla J &= \sum_{n} \varepsilon_n \nabla \varepsilon_n \\
\nabla\nabla J &= \sum_{n} \nabla\varepsilon_n \left(\nabla\varepsilon_n\right)^{\operatorname{T}}
+ \sum_{n} \varepsilon_n \nabla\nabla \varepsilon_n .
\end{align}
It is also worth noting here that we can write down the Taylor series expansion
(up to the second order) of a scalar function $f(\mathbf{x})$ compactly
in terms of the gradients as
\begin{equation}
f(\mathbf{x} + \Delta\mathbf{x}) \simeq
f(\mathbf{x})
+ \mathbf{g}^{\operatorname{T}} \Delta\mathbf{x}
+ \frac{1}{2} {\Delta\mathbf{x}}^{\operatorname{T}} \mathbf{H} \Delta\mathbf{x}
\end{equation}
where $\mathbf{g}$ and $\mathbf{H}$ are the gradient vector and the Hessian matrix of
$f(\mathbf{x})$, respectively, so that
\begin{equation}
\mathbf{g} \equiv \nabla f(\mathbf{x}) =
\begin{pmatrix}
\frac{\partial f}{\partial x_{1}} \\
\vdots \\
\frac{\partial f}{\partial x_{D}}
\end{pmatrix} , \quad
\mathbf{H} \equiv \nabla\nabla f(\mathbf{x}) =
\begin{pmatrix}
\frac{\partial^{2} f}{\partial x_{1} \partial x_{1}} &
\hdots &
\frac{\partial^{2} f}{\partial x_{1} \partial x_{D}} \\
\vdots & \ddots & \vdots \\
\frac{\partial^{2} f}{\partial x_{D} \partial x_{1}} &
\hdots &
\frac{\partial^{2} f}{\partial x_{D} \partial x_{D}}
\end{pmatrix} .
\end{equation}

\erratum{Page~698}
Equation~(C.20):
Although the Jacobian of a vector with respect to a vector is defined in (C.18),
the Jacobian of a matrix with respect to a scalar has not been defined.
The Jacobian~$\partial\mathbf{A}/\partial x$ of
a matrix~$\mathbf{A} = (A_{ij})$ with respect to a scalar~$x$ is
a matrix with the same dimensionality as $\mathbf{A}$
and is given by
\begin{equation}
\frac{\partial\mathbf{A}}{\partial x} = \left( \frac{\partial A_{ij}}{\partial x} \right)
\label{eq:Jacobian_of_matrix_wrt_scalar}
\end{equation}
which is analogous to (C.18) in that
the partial derivatives are laid out according to the numerator, i.e.,
$\mathbf{A}$ in \eqref{eq:Jacobian_of_matrix_wrt_scalar}.
On the other hand, the gradient~\eqref{eq:gradient_of_vector_wrt_vector} is such that
the derivatives are laid out according to the denominator.
We should also point out that,
in a similarly analogy, we write the gradient~$\nabla_{\mathbf{A}} y$ of
a scalar~$y$ with respect to a matrix~$\mathbf{A}$ as
\begin{equation}
\nabla_{\mathbf{A}} y = \left( \frac{\partial y}{\partial A_{ij}} \right) .
\label{eq:gradient_of_scalar_wrt_matrix}
\end{equation}

\erratum{Page~698}
Equation~(C.22):
For this identity to be well-defined,
it is necessary that we have $\operatorname{det}(\mathbf{A}) > 0$.
We should make this assumption clear.
Or, if we adopt the notation~\eqref{eq:abs_of_matrix_determinant} for $\left|\mathbf{A}\right|$,
we see that (C.22) holds for any nonsingular $\mathbf{A}$ such that
$\operatorname{det}(\mathbf{A}) \neq 0$.
The ``Eigenvector Equation'' section of Appendix~C gives us a hint for a proof of (C.22)
where $\mathbf{A}$ is assumed to be symmetric positive definite so that $\mathbf{A} \succ 0$.
Although the proof outlined in PRML is highly instructive,
we can give a more general proof where $\mathbf{A}$ is only assumed to be nonsingular,
which I would also like to have in PRML.
To this end, we first show the following identity for any square matrix~$\mathbf{A}$
\begin{equation}
\frac{\partial}{\partial x} \operatorname{det}\left(\mathbf{A}\right) =
\operatorname{Tr}\left(\mathbf{A}^{\dagger}\frac{\partial\mathbf{A}}{\partial x}\right)
\label{eq:Jacobi_formula}
\end{equation}
where $\mathbf{A}^{\dagger}$ is the adjugate matrix of
$\mathbf{A}$.
The $(ij)$-th element $A_{ij}^{\dagger}$ of the adjugate matrix~$\mathbf{A}^{\dagger}$ is given by
\begin{equation}
A_{ij}^{\dagger} = (-1)^{i+j} \operatorname{det}\left(\mathbf{A}^{(ji)}\right)
\label{eq:definition_of_adjugate}
\end{equation}
where $\mathbf{A}^{(ij)}$ is a matrix obtained by removing
the $i$-th row and the $j$-th column of $\mathbf{A}$.
From the identity
\begin{equation}
\mathbf{A}\mathbf{A}^{\dagger} = \mathbf{A}^{\dagger}\mathbf{A} =
\operatorname{det}(\mathbf{A})\mathbf{I}
\label{eq:adjugate_identity}
\end{equation}
we can write the inverse~$\mathbf{A}^{-1}$ in terms of the adjugate~$\mathbf{A}^{\dagger}$ so that
\begin{equation}
\mathbf{A}^{-1} = \frac{\mathbf{A}^{\dagger}}{\operatorname{det}(\mathbf{A})}
\end{equation}
if $\operatorname{det}(\mathbf{A}) \neq 0$.
The above identity~\eqref{eq:adjugate_identity} also implies
\begin{equation}
\operatorname{det}(\mathbf{A}) = \sum_{k} A_{ik} A_{ki}^{\dagger} .
\end{equation}
Substituting this into the left hand side of \eqref{eq:Jacobi_formula}
and noting that, from the definition~\eqref{eq:definition_of_adjugate},
$A_{ji}^{\dagger}$ is independent of $A_{ik}$ nor $A_{kj}$ for any $k$,
we have
\begin{align}
\frac{\partial}{\partial x} \operatorname{det}\left(\mathbf{A}\right) &=
\sum_{ij} \left\{ \frac{\partial}{\partial A_{ij}} \sum_{k} A_{ik} A_{ki}^{\dagger} \right\}
  \frac{\partial A_{ij}}{\partial x} \\
&= \sum_{ij} A_{ji}^{\dagger} \frac{\partial A_{ij}}{\partial x} \\
&= 
\operatorname{Tr}\left(\frac{\partial\mathbf{A}}{\partial x}\mathbf{A}^{\dagger}\right) =
\operatorname{Tr}\left(\mathbf{A}^{\dagger}\frac{\partial\mathbf{A}}{\partial x}\right) .
\end{align}
Having shown the identity~\eqref{eq:Jacobi_formula},
we can now evaluate the right hand side of (C.22) as
\begin{align}
\frac{\partial}{\partial x} \ln \left|\mathbf{A}\right| &=
\frac{\partial}{\partial x} \ln \left|\operatorname{det}\left(\mathbf{A}\right)\right| \\
&=
\frac{1}{\operatorname{det}\left(\mathbf{A}\right)}
\frac{\partial}{\partial x} \operatorname{det}\left(\mathbf{A}\right) \\
&=
\operatorname{Tr}\left(\mathbf{A}^{-1}\frac{\partial\mathbf{A}}{\partial x}\right)
\end{align}
where we have used the notation~\eqref{eq:abs_of_matrix_determinant} for $\left|\mathbf{A}\right|$
and assumed that $\operatorname{det}\left(\mathbf{A}\right) \neq 0$.

% TODO: Appendix C
% TODO: Appendix D

\erratum{Page~717}
Column 2, Entry 7:
``John Hopkins University Press'' should read ``The Johns Hopkins University Press.''

\bibliographystyle{chicago}
\bibliography{refs}

\end{document}
